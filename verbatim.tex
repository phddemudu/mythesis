
Learning from data is increasingly important in 21st century. We encounter data in our lives on daily basis mostly quantitative data such as prices of goods, weather, exam marks, sales, share prices , account balances, blood pressure reading, height, weight , body mass index and etc. Qualitative data can easily converted to quantitative data by coding.(Alternate Introduction)

Fortunately new fields of study such as Data Analyst, Data Science, Big Data and Data Analytic are also emerged for us to learn from these massive data.  

According to Chambers et al., " There is no single statistical tool that is as powerful as a well chosen graph". Often graphical summaries of data are very revealing and helpful in detecting outliers. One of the most commonly used and understood graphical summaries of values of numeric variables is the histogram. \cite{francis2005dancing}.

A histogram is useful to look at when we want to see more detail on the full distribution
of the data. The boxplot is quick and handy, but fundamentally only gives us a bit of
information. \cite{peng2012exploratory}

Why Graphics?
There is no single statistical tool that is as powerful as a well-chosen
graph. Our eye-brain system is the most sophisticated information
processor ever developed, and through graphical displays we can put
this system to good use to obtain deep insight into the structure of data.
An enormous amount of quantitative information can be conveyed by
graphs; our eye-brain system can summarize vast information qUickly
and extract salient features, but it is also capable of focusing on detail.
Even for small sets of data, there are many patterns and relationships
that are considerably easier to discern in graphical displays than by any
other data analytic method.\cite{john1983graphical}

A descriptive data analysis seeks to summarize the measurements
in a single data set without further interpretation \cite{leek2015elements}

An exploratory data analysis builds on a descriptive analysis
by searching for discoveries, trends, correlations, or relationships
between the measurements of multiple variables to
generate ideas or hypotheses. \cite{leek2015elements}

In a quality statistical data analysis the initial step has to be exploratory. This is
particularly true of applied data mining, which essentially consists of searching
for relationships in the data at hand, not known a priori.\cite{giudici2005applied}


More than 50 years ago, John Tukey called for a reformation of academic statistics. In `The
Future of Data Analysis', he pointed to the existence of an as-yet unrecognized science, whose
subject of interest was learning from data, or `data analysis'. Ten to twenty years ago, John
Chambers, Bill Cleveland and Leo Breiman independently once again urged academic statistics
to expand its boundaries beyond the classical domain of theoretical statistics; Chambers called
for more emphasis on data preparation and presentation rather than statistical modeling; and
Breiman called for emphasis on prediction rather than inference. Cleveland even suggested the
catchy name Data Science" for his envisioned field. \cite{donoho201750}
